\section{Installation}

This appendix briefly introduces how to install our XG-PON module
with NS-3. Here, we assume that NS-3 had been installed on your
computer and you are using the development code tree.

In this release of our XG-PON module, all files are put into one
folder "xgpon". In "xgpon", there are several folders that are
similar to other NS-3 modules, such as "bindings" (for Ptyhon
bindings), "doc" (for documentations included this manual),
"examples" (for examples to demonstrate how to use this module),
"helper" (for classes provided to researcher with aim to
facilitate its usage), "model" (fore source code), and "test" (for
test cases).

In "xgpon", we have another folder "changesOnOtherModules" that
contains some bug fixes related with other modules. Currently, we
just changed "ipv4-l3-protocol.cc" of the internet module to avoid
program crash when long simulations are carrying out.

To use this XG-PON module, you should first copy or link "xgpon" under
the folder "ns-3-dev/src". You should also copy
the above "ipv4-l3-protocol.cc" (within changesOnOtherModules) 
into "ns-3-dev/src/internet/model". Note that this file copy is necessary 
for NS-3.19 and the latest release of NS-3 may have already fixed this bug.
You may compare the function "ProcessFragment()" of the latest version 
with our copy to make sure that the bug has been fixed in the latest NS-3 release.
After that, please configure and build NS-3 as normal. XG-PON module
should be ready to be explored.

In addition, in "xgpon", there is one folder "scripts4evaluation"
that holds the scripts used by us to evaluate this module. Before
running these evaluations, you need copy the corresponding script
file into "ns-3-dev". To test GiantMAC through running 
"pedro/pargiantscript.py"), you need set one new environment
parameter (NS3) according to the path of NS-3 on your system. 
You should also create "data/giant-data" folder in your NS-3
installation.
