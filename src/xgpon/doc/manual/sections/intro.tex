\section{Introduction} \label{section_introduction}

During the last few decades, we have witnessed the huge success of
the Internet, which has changed our daily life significantly and
has become one of the main economy engines. In these years, the
infrastructure of the Internet kept evolving to provide better
performance, and optical communication is one of its driving
forces.

%Core networks of the Internet
%have evolved into all-optical networks since optical fibers can
%provide huge amount of bandwidth through the matured DWDM (Dense
%Wavelength Division Multiplexing) technology.
Transmission links in the network core are already based on
optical fiber technology, which provides huge amount of bandwidth
through the matured DWDM (Dense Wavelength Division Multiplexing)
technology.
%Instead of Mbps or
%Gb/s, the bandwidth of a core network is now measured in Tbps
%(Tera-bps: 1000 Gb/s) and it can provide almost un-limited bandwidth.
%For example, bandwidth capacity of the new Trans-Pacific Express is 5.12 Tb/s. \footnote{http://en.wikipedia.org/wiki/TPE\_(cable\_system)}.
More recently, optical fibers have also found their application in
access networks to provide high speed Internet access to end
users.
%Besides core networks, optical fibers have also been deployed in
%access networks to provide high speed Internet access to end
%users.
FTTx (Fiber To The Home/Building/Curb, etc.) networks based on
Passive Optical Network (PON) technologies, such as
Gigabit-capable PON (GPON) standardized by the Full Service Access
Network (FSAN) group of the International Telecommunications Union
(ITU) \cite{itu08gpon} and Ethernet PON (EPON) standardized by the
Ethernet in the First Mile (EFM) task force of the Institute of
Electrical and Electronics Engineers (IEEE)
\cite{ieee04EPON8023ah}, have been widely deployed in many
countries such as the US, Korea and Japan.

%However, compared to ADSL and
%Cable that could exploit the existing wires, the much higher
%capital expenditures hindered the deployment of FTTx in many areas.

%Long Reach PON (LR-PON) has emerged recently as a feasible
%solution (in terms of both technology and economy) to bring
%optical fiber into access networks and provide high bandwidth to
%its users \cite{payne02lrpon}\cite{shea07lrpon10G1000onu}. Through
%increasing the bandwidth and extending the reach of
%current PON technology to 100 km, the network can be consolidated further
%and one optical fibre can be shared by more users.

10-Gigabit-capable Passive Optical Network (XG-PON) is a new
standard released by the FSAN that improves G-PON, by increasing
the default downstream data rate to 10 Gb/s, while increasing the
upstream data rate to 2.5 or 10 Gb/s. Also, the maximum number of
users per wavelength is increased from 64 to 256, and amendments
are being defined for extending the physical reach up to 60 Km.


%Consequently,
%LR-PON can reduce both the capital expenditure and operational
%expenditure significantly.

%With the maturity of PON technology, LR-PON concepts have been
%incorporated into the G-PON standard.
%the key concepts of LR-PON into GPON has been released by the FSAN
%as the standard for the next generation optical access networks
%\cite{itu10XGPON}.

%Ten Gigabit Passive Optical Network (XG-PON) improves the older GPON standard,
%by increasing the default downstream data rate to 10 Gbps,
%while upstream it can vary between 2.5 and 10 Gbps.
%Also the number of users is increased from 64 to 256,
%and amendments are being defined for extended reach up to 60 Km.

%Our ultimate goal is to study Long-Reach PON (LR-PON),
%an evolution of XG-PON, with a larger number of users,
%symmetric data rate (i.e., 10Gbps upstream and downstream),
%and longer reach (100+km). Our aim is to initially build LR-PON from the XG-PON standard,
%while identifying required modification and improvements.

Since XG-PON could pave the way for many bandwidth-intensive
applications (IPTV, Video On Demand, Video Conference, etc.), it
is very important to study the performance issues arising with the
deployment of XG-PON. For instance, it is valuable to study the
impacts on the performance of XG-PON, when the propagation delay
is much longer than that of the current PON networks
\cite{song09multithread4LRPON}. It is also important to
investigate the interactions between XG-PON and the upper-layer
protocols (TCP \cite{postel81tcp}, etc.) for improving user
experience \cite{ikeda09tcpPON}. In addition, XG-PON has been
proposed for Fiber To The Cell, in which XG-PON acts as the
backhaul for multiple base stations of a cellular network
\cite{itu10XGPON}. Under this scenario, it is also very valuable
to study its integration with various wireless networks (LTE
\cite{lte}, WiMAX \cite{ieee04wimax}, etc.) for providing high
speed mobile Internet access.

Considering that XG-PON is still in its early stage, the above
research topics should be first studied through simulation since
it is too expensive to setup one XG-PON testbed and it is too
complex to model the above scenarios to be studied with enough
details. In this report, we present an XG-PON module for NS-3
\cite{ns3}, a state of the art open-source network simulator. Our
XG-PON module is based on a series of G.987 Recommendations from
the FSAN group of ITU. These recommendations mainly define the
specifications of Physical Media Dependent (PMD) and Transmission
Convergence (TC) layers of XG-PON. To study the above research
topics with reasonable simulation speed, the optical distribution
network and the operations of physical layer are simplified
significantly. This XG-PON module focuses on the issues of TC
layer, such as frame structure, resource allocation, Quality of
Service (QoS) management, and Dynamic Bandwidth Assignment (DBA)
algorithms for the upstream wavelength. During the design and
implementation of this module, we have also paid a lot of
attention on its extensibility and configurability. Since this
XG-PON module needs to simulate a 10Gbps network and hundreds of
ONUs, its performance (the speed of simulation and the overhead of
memory) has also been given a high priority when designing and
implementing these components.

This XG-PON module is built completely in C++ with 72 classes and
approximately 22,000 lines of code. These code are under the GNU
General Public License and can be downloaded from our websites
(CTVR\footnote{\url{www.ctvr.ie}} and
MISL\footnote{\url{http://www.ucc.ie/en/misl/}}). To the best of
our knowledge, it is the first XG-PON module for NS-3. We believe
that this work is a significant contribution to the scientific
community as it allows to simulate XG-PON, the next generation
optical access network, and study the performance issues that
arise with the deployment of XG-PON.


This report is organized as follows. Section
\ref{section_bg_related} briefly introduces PON, NS-3, and related
works. The details of XG-PON are then presented in section
\ref{section_xgpon}. The design principles and key decisions are
discussed in section \ref{section_choices}. The important
trade-offs made by us in terms of simulation accuracy and
simulation speed are also discussed throughout. Section
\ref{section_design} presents the details of the design and
implementation of our XG-PON module for NS-3. The preliminary
evaluation results are then presented in section
\ref{section_results}. Finally, section \ref{section_conclude}
concludes this report with several directions for future work. At
the end of this report, three appendices are also attached for
introducing the installation with NS-3, the source files, and one
example of XG-PON simulation.
