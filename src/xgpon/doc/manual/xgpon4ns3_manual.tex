%\documentclass[conference]{IEEEtran}
%\documentclass[peerreview]{IEEEtran}
%\documentclass[12pt, onecolumn, draftcls]{IEEEtran}
\documentclass[onecolumn, draftcls]{IEEEtran}

\usepackage{amsmath}
\usepackage{graphicx}
\usepackage{subfigure}
\usepackage{listings}
\usepackage{color}
\usepackage[small,bf]{caption}
\usepackage{url}
\usepackage{latexsym}
\usepackage{lscape}

\begin{document}


\title{An XG-PON Module for the NS-3 Network Simulator: the Manual}

%\numberofauthors{2} 

\author{
% You can go ahead and credit any number of authors here,
% e.g. one 'row of three' or two rows (consisting of one row of three
% and a second row of one, two or three).
%
% The command \alignauthor (no curly braces needed) should
% precede each author name, affiliation/snail-mail address and
% e-mail address. Additionally, tag each line of
% affiliation/address with \affaddr, and tag the
% e-mail address with \email.
%
% 1st. author

Xiuchao Wu, Kenneth N. Brown, Cormac J. Sreenan, Jerome Arokkiam\\
 Department of Computer Science, University College Cork, Ireland\\
\{xw2, k.brown, cjs\}@cs.ucc.ie, \{jerome\}@4c.ucc.ie \\\vspace{0.2in}
% 2nd. author
 Pedro Alvarez, Marco Ruffini, Nicola Marchetti, David Payne, Linda Doyle \\
CTVR / The Telecommunications Research Centre, Trinity College Dublin, Ireland\\
\{pinheirp, marco.ruffini, marchetn, ledoyle\}@tcd.ie, david.b.payne@btinternet.com 
}


\maketitle \thispagestyle{empty}

\begin{abstract}

10-Gigabit-capable Passive Optical Network (XG-PON), one of the
latest standards of optical access networks, is regarded as one of
the key technologies for future Internet access networks. This
report presents our XG-PON module for the NS-3 network simulator.
This module is designed and implemented with aim to provide a
standards-compliant, configurable, and extensible module that can
simulate XG-PON with reasonable speed and can support a wide range
of research topics. These include analyzing and improving the
performance of XG-PON, studying the interactions between XG-PON
and the upper-layer protocols, and investigating its integration
with various wireless networks. In this report, we will introduce
PON and XG-PON, discuss design principles and trade-offs made
during the course, describe the design and implementation details,
and present the preliminary evaluation results.


\end{abstract}



\input sections/intro.tex
%%%introduce optical fiber in the last mile and the promising LR-PON briefly;
%%%present the motivations for this LR-PON module and summarize our work;
%%%paper organization.


\input sections/background.tex
%%%introduce PON, LR-PON with some details.
%%%research opportunities with LR-PON
%%%related work
%%%briefly introduce NS-3 and the motivations to implement a LR-PON module for NS-3


\input sections/xgpon.tex
%%%details of XG-PON.

\input sections/choices.tex

\input sections/design.tex
%%%challenges: standard issue, simulation speed, configuration, extensibility
%%%design principles adopted in this module
%%%simulation choices
%%%Design of this module
%%%implementation status
%%%preliminary evaluation, such as validation and simulation speed ({\color{red} depend on coding progress})


\input sections/results.tex


\section{Summary and Future Work} \label{section_conclude}


In this report, we introduce an XG-PON module for the NS-3 network
simulator. We describe the details of its design and
implementation, and present some preliminary evaluation results.
These results indicate that our XG-PON module is quite robust and
can simulate XG-PON with reasonable speed and moderate memory
consumption. As the first XG-PON module for NS-3, we believe that
this work is a significant contribution to the scientific
community as it allows us to simulate XG-PON and study the
performance issues that arise with the deployment of XG-PON.

In the future, we will implement more scheduling and DBA
algorithms that were proposed for G-PON or XG-PON. We will also
keep improving its simulation speed and parallel/distributed
simulation will be considered. Furthermore, we will study how to
simulate Fiber to the Cell with this XG-PON module and the
WiMAX/LTE modules distributed with NS-3. The potential performance
issues mentioned in this report will also be investigated using
this XG-PON module.




%ACKNOWLEDGMENTS are optional
\section{Acknowledgments}
This work is supported in part by Science Foundation
of Ireland through CTVR (http://www.ctvr.ie/).

\appendices
\newpage
\input sections/installation.tex
\newpage
\input sections/files.tex
\newpage
\input sections/example.tex



%-------------------------------------------------------------------------
\bibliographystyle{IEEEtran}
\bibliography{IEEEabrv,refxcwu}



\end{document}
